\begin{multicols}{2}
ции на сфере. Интересно отметить, что "аксиомами" I' - V' оно определяется однозначно.)

\indent Пусть $\vec{p}$ - вектор. Введем вспомогательную функцию $p$ следующим образом: для любой точки $A$ нашей сферы обозначим через $p(A)$ проекцию вектора $\vec{p}$ на ось, определяемую вектором $\vec{OA}$.

\indent Рассмотрим среднее значение $M$ $(|p|)$ функции $A \longrightarrow |p (A)|$ на сфере. Покажем существование такого $k \neq 0$, что для любого вектора $\vec{p}$ выполняется (6). Для этого, ввиду свойства II', достаточно доказать, что $|\vec{p}| = |\vec{q}|$. Обозначим через $R$ какой-нибудь поворот пространства вокруг оси, проходящей через точку $O$, который переводит луч с направляющим вектором $\vec{q}$ в луч с направляющим вектором $\vec{p}$. Тогда для любой точки $A$ сферы $q(A) = p(R(A))$. Из V' вытекает $M(|q|) = M(|p|)$.

\indent Дальнейнее решение задачи 3 дословно повторяет решение задачи 2. (Упражнение 5 и задачи 2, 3 исчерпывают задачу M394.)

\subsection*{4. Длина через ширину}
\indent Идею, на которой основано решение задач 2 и 3, можно использовать для вычисления длины плоской замкнутой выпуклой ломаной. 

\indent Пусть $\textup{Л}$ - такая ломаннная, $a_1,~a_2,~...,~a_n$ - ее звенья. Фиксируем некоторую ось $l_0$. Пусть $l_\alpha$ - ось, образующая с осью $l_0$ угол $\alpha$. Обозначим через $Ш(\alpha)$ "ширину" нашей ломаной в направлении оси $l_\alpha$, т. к. длину ее проекции на ось $l_\alpha$. Оказывается, если знать "ширину" ломаной $Л$ в произвольном направлении, т. е. уметь вычислять "ширину" ломаной $Л$ в произвольном направлении, т. у. уметь вычислять функцию $\alpha \xrightarrow{} Ш (\alpha)$, то можно найти ее длину L. Покажем, как это сделать.

\indent Обозначим через $a_i (\alpha)$ длину проекции звена $a_i$ на ось $l_a$.

\example {Упражнение 8}. Докажите, что $\textup{Ш}(\alpha) = \frac{1}{2}[a_1(\alpha) + a_2(\alpha) + ... + a_n(\alpha)$
\indent В решении задачи 2 было показано, что среднее значение функции $\alpha \xrightarrow{} \a_i(\alpha)$ пропорционально $|a_i|$. Из (5) и (7) коэффициент пропорциональности равен $\frac{1}{2\pi} \int\limits_0^{2\pi}$. Из упражнения 8 и свойств I', II' среднее значение $M(Ш)$ функции Ш равно полусумме средних значений  функций $\alpha \xrightarrow{} \a_i(\alpha)$. Следовательно, 
\[M(\textup{Ш}) = \frac{1}{2} (\frac{2}{\pi} |a_1|\]
\[\indent + \frac{2}{\pi} |a_2| + ... + \frac{2}{\pi} |a_n|) =\]
\[= \frac{1}{\pi} (|a_1| + |a_2| + ... + |a_n|) = \frac{1}{\pi} L\]

Отсюда и из (2)
\[L = n*M(\textup{Ш}) = \pi * \frac{1}{2\pi} =\]
\[=\frac{1}{2}\int\limits_0^{2\pi}\textup{Ш}(\alpha)~d\alpha \indent(9)\]

Таким образом, знаю функцию Шб мы можем найти длину L ломанной Л.

\example{Упражнение 9}. Докажите, что если длины всех сторон и диагоналей выпуклого многоугольника меньше $d$, то его периметр меньше $\pi d$. 

\indent Формула (9) справедлива для любой плоской замкнутой выпуклой кривой. Изложенный метод определения длины "через ширину" предложил в 1930 году известный польский математик Г. Штейнгауз.

\subsection*{5. Длина суммы}
\example{Задача 4}. На плоскости даны векторы $\vec{a_1},~\vec{a_2},~...,~\vec{a_n}$, сумма длин которых равна 1. Докажите, что среди них можно выбрать несколько векторов, длина суммы которых не меньше $\frac{1}{\pi}$

\indent Решите эту задачу, следуя предлагаемому ниже плану. Пусть З - подмножество множества ${\vec{a_1},~\vec{a_2},~...}$

\end{multicols}
